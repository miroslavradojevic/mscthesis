\chapter{Problem Definition} \label{chap:problem-def}
Robots are designed to help humans by replacing them or helping them in accomplishing certain task. This observation stands for the underwater robots in their mission of exploring vast environment such as water. It is essential for successful application of an underwater vehicle that the vehicle can accurately estimate its own position in the environment. It autonomously makes decisions that are influenced by position information. Moreover, usability and quality of the data corresponds to the precision of the localisation.
\section{Robot localization}
With the discovery and the development of Global Positioning System (GPS), the issue of the localisation of all robots operating on land or in the air was solved with the considerably cheap and reliable system. However, designers of those machines that operate underwater remain being in quest for the localization system of similar performance that would work in vast and unique environment such as water. Due to absorption of radio frequencies in the salty water, GPS signal is not available in the sea depths, and radio-wave-based localization which serves as standard ``above the surface'' cannot be used. As a consequence, various methods are developed to establish the localization underwater using all potentially useful sensor information. 

The improvements of inertial instruments' performance enabled their usage in localization for calculating the odometry. Different types of sensors are typically used and the role of mathematical algorithm such as Kalman filter or its variations in this application would be to combine together sensor measurements from all the different sources and make an optimal estimate of the vehicle state: its position and orientation. 

The aim of the project is to implement and evaluate navigation algorithm for an underwater vehicle. Navigation, as introduced in literature, implies two capabilities \cite{farrell98}:
\begin{itemize}
\item localization - accurate determination of the vehicle position and velocity with respect to a known reference point
\item planning and the execution of the movements between locations   
\end{itemize} 
The work presented in the thesis will focus on the first capability. Moreover, the first capability is a necessary step in carrying out the second capability correctly. To accomplish the task, sensor information is integrated in calculations. The purpose is to calculate the vehicle position in every moment as accurately as possible. Mathematical tool that will integrate the measurements and establish the final estimate is well known Kalman filter \cite{kalman60} algorithm. More details about Kalman filter in Sections \S~\ref{sec:kf} and \S~\ref{sec:ekf}. 
%Localisation is a crucial problem to solve for practical applications - for instance successful dam inspection mission requires precise navigation system so that the position with respect to the dam can be known all the time \cite{blain03} thus giving the meaning to the robot activity.

It is important to mention that localization underwater tends to be a different challenge compared with the localization on land or in the air. Namely, the usage of GPS signal (absolute position information) is limited since it is available only on water surface. Therefore, speed and heading information obtained from inertial navigation system (INS) sensors is used together with the mathematical model of the motion to calculate the position, orientation and velocity via dead reckoning. Such strategy eventually leads to progressive error since measurement errors are integrated each time. To overcome this, GPS information which gives absolute position is used to make corrections. It updates the position information whenever available - either directly, if the vehicle is on the surface or using long baseline acoustic positioning (LBL).

Another obstacle in managing localization is the water environment itself. Usage of sensors based on light transmission such as camera, is limited because environment is such that it deforms or decreases the signal. A useful tool to determine the distance or the speed in water is sound, a mechanical wave that does not severely depend on light conditions and moves faster in water.  

\T{Issues: } Localisation algorithm starts with the initial estimate of the location and continues making estimation by processing available information obtained from measurements. How the information from several sensors can be combined together used to estimate vehicle position? Measurements are of various kinds, therefore, one of the problems to solve on the way to having a successful localisation is the integration of those different sensors together into one unique representation of the environment \cite{negenborn03}. Such problem is recognized as \textit{multi - sensor fusion}. It is certainly one of the crucial tasks to solve in navigation of an underwater vehicle. Localisation algorithm is intended to take the input from different sensors and calculate the best possible estimate of the vehicle position. 

Having measurements is a necessity, but having noisy or even false measurements or vehicle states is a problem to deal with. Kalman filter is intended to treat the signal as being a random variable or a function of a random variable with Gaussian distribution. Standard deviation of random variable manifests its uncertainty which compensates for the role of the noisy state or measurement.

Navigation of underwater robot is vital part of inspection and mapping tasks. Large submarines use precise measurements taken from high-priced sensory devices. Future seems to involve more and more usage of small, less expensive systems, that tend to fuse together information from variety of inexpensive, less accurate sensors in order to improve the quality of the navigation. 