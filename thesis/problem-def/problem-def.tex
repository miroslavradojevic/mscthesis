\chapter{Problem Definition} \label{chap:problem-def}

Robots are designed to help humans by replacing them or helping them in accomplishing certain task. This observation implies underwater robots in their mission of exploring vast environment such as water. It is essential for successful application of an underwater vehicle that the vehicle can accurately estimate its own position in the environment.  

First of all, it autonomously makes decisions that are influenced by position information. Second, usability and quality of the data corresponds to the precision of the localization.

With the discovery and the development of Global Positioning System (GPS), the issue of the localization of all robots operating on land or in the air was solved with the considerably cheap and reliable system. However, designers of those machines that operate underwater remain being in quest for the localization system of similar performance that would work in vast and different environment such as water.

Due to the absorption of radio frequencies in salt water, GPS signal is not available in sea depths, and radio localization which serves as standard ``above the surface'' cannot be used. Therefore, various methods are developed to establish the localization underwater. New ways of measuring the vehicle position have been used  The aim of the thesis report is to  

The improvement of sensor performance enabled their usage in localization. Different types of sensors are typically used and the role of mathematical algorithm such as Kalman filter or its variations is to combine together sensor measurements from all the different sources and make an optimal estimate of the vehicle state: its position and orientation. 

The aim of the project is to implement and evaluate navigation algorithm for an underwater vehicle. Navigation, as introduced in literature, implies two capabilities \cite{farrell98}:
\begin{itemize}
\item localization - accurate determination of the vehicle position and velocity with respect to a known reference point
\item planning and the execution of the movements between locations   
\end{itemize} 
The work presented in the thesis will emphasize the first capability. Moreover, the first capability is a necessary step in carrying out the second capability correctly. To accomplish the task, sensor information is integrated in calculations. The purpose is to calculate the vehicle position in every moment as accurately as possible. Mathematical tool that will integrate the measurements and establish the final estimate is well known Kalman filter \cite{} algorithm. More details about Kalman filter in chapter \label{chap:KalmanFilter}. 

It is important to mention that localization under water tends to be different challenge compared with the localization on land or in the air. Namely, the usage of GPS signal is limited since it is available only on water surface. Therefore, speed and heading information obtained from inertial navigation system (INS) sensors is used together with the mathematical model of the motion to calculate the position, orientation and velocity via dead reckoning. Such strategy eventually leads to progressive error since measurement errors are integrated each time. To overcome this, GPS information which gives absolute distance is used to make corrections. It updates the position information whenever available - either directly, if the vehicle is on the surface or using long baseline acoustic positioning (LBL).

Another obstacle in managing localization is water environment itself. Usage of sensors based on light transmission such as camera, is limited because environment is such that it deforms or decreases the signal. A useful tool to determine the distance or the speed in water is sound, a mechanical wave that does not severely depend on light conditions and moves faster in water.  