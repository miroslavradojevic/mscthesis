\section{Related work on AUV navigation}\label{sec:lit-review}

Following paragraphs summarize the documented ways to process the sensor information in order to be able to estimate the position within the environment.

In existing survey on underwater vehicle navigation, Kingsey \cite{kinsey06} gives a summary presenting the methods used for navigation. As introduced in \cite{kinsey06}, current vehicle position is named navigation state - a vector whose elements express where the vehicle is and how it is oriented in space. Localization simply means finding a way to estimate navigation state vector. Naturally, sensors provide the data for the estimation.

The most simple approach would be dead reckoning - to take the raw sensor measurements and use them directly or within a simple mathematical model that describes the vehicle dynamics. However, many techniques presented in literature utilize sensor data as supplementary information with the information from the kinematic model.

Underwater navigation is using several instrumentation methods to carry out the robot localization in the sea \cite{whitcomb99}. These include transponder networks placed on the bottom of the sea, tracking systems between the ship and the underwater vehicle, and sensoring devices that measure range and dynamics mounted on the vehicle itself. Each of the methods has advantages and disadvantages. Transponder network gives accurate position information, however using it means installing and calibrating additional equipment \cite{eustice05}.

Some available works have already delt with the issue of managing localization using Kalman filtering. Master thesis of Negneborn \cite{negenborn03} gives a useful overview of the theoretical knowledge and surveys the utilization of Kalman filtering for localizing underwater vehicles. 

Blain et al. \cite{blain03} study the application of Kalman filter in navigation of an underwater vehicle used for water dam inspection focusing on merging position and velocity information. This algorithm uses acoustic positioning sensor together with the integration the measured DVL sensor velocities \cite{blain03} to estimate the position. Several issues are delt with in their work. 

Kalman filter output could be corrupted in situations when observations (sensor measurements) are subject to interruption or periodic stopping. Due to the fact that not all the sensors can be always available, it is usually necessary to be capable of adding or removing sensor observations from the system without changing the navigation algorithm. 

Asynchronous data delivery in this particular case means that the DVL sensor provides data with higher rate \cite{blain03}. Such obstacle was solved by switching to estimation procedure suitable for that particular sensor measurement scenario. This simply means that if the acoustic sensor and DVL sensor provide new measurement, Kalman filtering is used to carry out the fusion. Otherwise, pure DVL velocity measurement is just integarated to update the position, as dynamic model would suggest. 

Delays in measurements are evident in case of acoustic measurements where the time of the measurement (timestamp) is current measurement time minus the time it took for the acoustic signal to arrive. To overcome this, position estimate between two acoustic measurements is memorized. Position is meanwhile updated by integrating the DVL data. The procedure consists of two stages: at first, we make a new position estimate obtaine by integrating veolcities from the new position estimate to the actual time - correction of the position estimate. 

Drolet et al. \cite{drolet00} introduce a flexible localization strategy based on sensor fusion and usage of several Kalman filters arranged together in a bank. Each filter is reduced to express simple cinematic equation and processes one state - works in one dimension. Idea is to integrate together sensor measurements that arrive at different time moments from different sensors. Method takes asynchronous information from sensors, manages a filter switching process so that the most recent data is used to update those filters that can be updated with such measurement \cite{drolet00}. Such sensor fusion strategy is adaptable in terms of number of sensors so that the best is taken out from the available input data, more robust to data loss. Moreover, asynchronous inputs are allowed. 

Di Massa et al. report usage of Kalman filter framework for slightly different concept of navigation that takes surrounding terrain as reference for esitmating the position of the sonar (``terrain-relative navigation''), \cite{diMassa97}. In their work, sonar image is matched to the map using mean absolute difference (MAD) as the matching criterion. Matching map location is considered as measurement of vehicle position \cite{diMassa97}. Several matchings are selected and weighted depending on how much they relate to the terrain images. Weights correspond to uncertainties in estimation theory. Quality of similarity is used to weight each measurement. Solution consists of having resulting best estimate of location \cite{diMassa97}. Information from selected matches is combined to make the best estimate. The role of Kalman filter framework is to carry out the estimation. Each of the chosen mathes is considered as one measurement together with its weight as uncertainty. The filtered state is the position of the image within the map. 

Gade and Jalving introduce aided post processing navigation system deployed on a commercial underwater vehicle \cite{gade99}. Idea is that vehicle records sensor data while accomplishing mission under the sea surface. At the same time, a vessel is positioned on the surface recieving information of its position through the reliable Differential Global Positioning System (DGPS). After the mission is over, data are combined together with position data that was simultaneously recorded on the survey vessel located on the surface. Kalman filtering is used when merging the data. \textit{Error-state} Kalman filter is used to combine sensor measurements and their error models. Observations in case of such filter is the difference between measured and computed values. Instead of working directly with states, presented algorithm filters the errors, so that the ultimate position and heading estimate can be derived by subtracting the estimated erros from, as authors suggest, corresponding calculated state elements. This way, final aim of obtaining vehicle position and heading together with the accuracy of such estimate \cite{gade99}. 

Dissertation \cite{roman05} gives the suggestion how to improve the vehicle position estimation when reconstructing maps of the sea floor. Visual information of the terrain is used as the feedback that makes terrain mapping data and the vehicle navigation data more consistent. Inspiration for investigating lies in fact that map-making depends on localization quality. Navigation errors are potentially large scale particularly seriously affecting the results when mapping is vehicle-based \cite{roman05}. Existing local navigation is used together with terain-relative measurements. Namely, terrain sub-maps are created over short periods while the vehicle works out the innacurate localization using dead reckoning. Sub-maps are registered resulting in position measurements between two vehicle states, placing an additional constraint on the vehicle positon estimates. Delayed EKF is used to merge together the measurements (``sub-map'' registrations and previously reached vehicle locations) into the navigation framework. Delayed state version of the recursive EKF enables retaining knowledge of prior platform positions.  

Yun et al. introduce and present simulation and testing results of the navigation system that combines the usage of Inertial Measurement Unit (IMU) together with GPS fixes that occur less frequent and asynchronously \cite{yun00}. Asynchronous Kalman filter with six states for orientation and eight states for position estimation is implemented \cite{yun00}. Process model takes the velocities and GPS bias, models them as white noises passed through the first order systems with the time constant. Measurement consists of synchronous velocity measurements and asynchronous DGPS information. The design of the filter for the position estimation algorithm conforms to the standard routine, with the difference that the measurement vector has different length depending on the number of available valid sensor inputs, hence it has a flexible size, but each observation updates the state vector of the fixed size \cite{yun00}. The idea of the asynchronism is that DGPS signals are used, if available and as soon as they are available, together with the speed measurements. This way, the localization algorithm uses the most of the data that are currently available.

The usage of the stochastic estimators implies having a known model that describes system state transiton from one moment to another (plant model) and model that desribes transiton from state to the measurement (observation model). Such model does not have to be the same each time. Jakuba and Yoerger \cite{jakuba03} study the way to optimize navigation by estimating the vehicle model parameters, for instance various dynamics or buoyancy coefficients that normally influence the model, but are treated as constant. Their study involves postprocessing of the navigation data and heuristic estimate of these coefficients' optimal value. Real missions that applied the technique resulted in reduced noise in localization data, therefore giving clearer tracking. 

Julier and Uhlmann introduce the method that carries out nonlinear filtering \cite{julier96}. It is an alternative generalization of the KF that changes the approach of representing mean and variance of the random variable. Their research is a quite useful and comprehensive theoretical overview of filtering in general and the role of Extended Kalman Filter (EKF) in switching to nonlinearity world. Introduced filter, later known as Unscented Kalman Filter (UKF) is regarded as more precise alternative to EKF that is, in addition, easy for implementation. More theoretical details about UKF are given in chapter \S~\ref{chap:kalman}.    Julier and Uhlmann point out the failigs of the EKF. In search of general method that would overcome the problem, instead of using proposed equations for projecting mean and covariance, a discrete set of points is chosen and projected using a chosen non-linear transformation. Idea is to use the parameters to approximate the Gaussian distribution instead of approximating the nonlinear transformation. This way, propagation of the information is accomplished directly, and the aim is to find a way to parametrise the information about the mean and covariance of the distribution. Advantages of the filtering algorithm are in terms of precision and simplicity (no need for Jacobian derivation) with empirical results for highly nonlinear problems including vehicle control indicating as good as or better performance than EKF and higher robustness.

Wan and van der Merwe \cite{wan00} go further in exploring the concept of UKF introduced by \cite{julier96}. Their research birefly reminds of disadvantages manifested in EKF and improvents gained with the usage of UKF. UKF theoretical backgrounds, the idea itself and meaning of used variables were explained in comprehensive manner in one of the document sections. Usage of UKF was reported together with the results in different estimation problems such as nonlinear system identification, state estimation, parameter estimation and dual estimation problems \cite{wan00}. UKF according to the authors achieved higher accuracy compared with EKF in all the domains that were examined.

Monte Carlo methods, based on repeated random sampling when computing the results are covered in several works, mostly dealing with Particle Filters (PF). Gordon et al. enclose \textit{bootstrap filter} \cite{gordon93}, later known as Particle Filter (PF) - a recursive algorithm based on representing state vector as set of random samples which are updated and propagated. Update stage of such algorithm uses Bayes rule, however the sampling strategy implies that the state space grid is not necessary the samples are localized in regions of high probability density \cite{gordon93}. Arulampalam et al. review Bayesian algorithms for nonlinear or non Gaussian problems. The emphasis of the review is on Particle Filters (PF), their features, variants and finally inevitable comparison with the standard EKF \cite{arulampalam02}. Doucet in his book on Monte Carlo methods \cite{doucet01} focuses on creating an broad summary of theory and various applications of bootstrap filters, optimal Monte Carlo filters and Particle Filters. Common feature of both ``Unscented'' and ``Monte Carlo Method'' estimation techinques is the sampling phenomenon. By using sampling, linearization of plant and obesrvation models is avoided, hence the cause of approximation error that existed in EKF-based methods is cancelled this way. PF can handle non Gaussian and and nonlinear processes, particualrly exhibited in AUV models. Moreover, PF does not need to have the initial information about the state. Sampling techniques, particularly PF, have been recently and increasingly applied as tool for navigation of an undewater vehicle.

Karlsson et al. study a sea navigation method that relies on underwater maps (depth map) and sonar measurements that support the navigation system \cite{karlsson02}. Particle Filter is used for state estimation. Since the problem of underwater navigation using depth map is nonlinear, sequential Monte Carlo methods are used, therefore state probability density is approximated with set of partices where each particle has a location and weight assigned to it. Both values reflect the value of the density of the region in the state space \cite{karlsson02}. Hence, instead of updating mean and covariance of the state, particle location and the weight of each particle are updated with each observation using sampling importance resampling (SIR) algorithm \cite{karlsson02}, \cite{gordon93}. Prior to navigation, terrain map (reference) was created using sonar depth measurements together with Differential GPS measurements and the obtained grid was used for navigation. Moreover, the usage of Cram\'{e}r-Rao bound was investigated in tasks like INS system design, sensor performance or even the amount of control of that is needed for the navigation. This work presents a succesful appllication of particle filtering for underwater navigation.

Above-mentioned work of Di Massa \cite{diMassa97} presents navigation guided by the depth measured with bathymetric sonar. Map-matching with digital bathymetric map stored on-board has been accomplished using the Probabilistic Data Association Filter (PDAF) - a recursive algorithm similar to Kalman filter, designed for one target of interest and several measurements of the tagret state available each time step \cite{diMassa97}. Position within the bathymetric map was stored as the state vector that was filtered. Results proved that such navigation is possible and that having a more diverse sea floor leads to more accurate navigation.

Maurelli et al. \cite{maurelli08} propose a particle filter underwater vehicle localization method in both structured and unstructured environment that is known prior to localization without having information about initial position and orientation of the vehicle. The work explores the possibility of dealing with dynamical situations when carrying out the localization and contributes in improving the particle filter algorithm. Improvements concern computational efficiency and effective way of treating state space in order to recover from wrong convergence when using PFs. Simulation and real experimental results are available.

%%% GPS AIDED %%%%%%
Caiti explored localization technique that uses floating acoustic buoys provided with GPS connection \cite{caiti05}. Idea is that buoys supply the vehicle with their GPS location by emitting the information at regular time intervals. This way, vehicle can calculate the time of flight of the acoustic signal and locate itself with respect to the buoys. In such constellation, additional equipment has to be installed and maintained. Furthermore, acoustic signals are not reliable, their range is limited and signals not alway available. 

Erol \cite{erol07} proposes a method for localization of the network of underwater sensors using single AUV as aiding device. This is just one of the examples of utilization of knowledge on AUV location. The aim is to use it to maintain localization of group of other objects in the water such as acoustic sensors. It is a system where AUV initially and occasionally receives GPS signals while being on the surface. Once the GPS location is received, vehicle dives to a certain depth and follows the defined path in between the sensor network. Set of freely deployed acoustic sensors is receiving messages containing coordinates from the vehicle, since the vehicle maintains updating its position using dead reckoning combined with occasional GPS correction. Emphasis is on algorithms for distance estimation so that proper values for sensor coordinates can be passed on to sensor network localization algorithm.

%%%%%%%%% USING A-PRIORI MAP %%%%%%%%%%%%
Eustice experiments with re-navigation for AUVs \cite{eustice05towards}. The aim is to use a-priori given, ship derived bathymetric maps to reduce dead reckoning drift by comparing ship-derived depth map with the depth map created by vehicle. The difference between them is used as correction, a tool for removing long-term drift. 

Williams is presenting a new method that uses terrain features for aiding the tracking of underwater vehicles in unstructured environments \cite{williams06}. Benefit of such research lies in creating a vehicle capable of adopting to terrain changes therefore capable of being reliably deployed for longer time deep underwater, on a real task, with real environment. This work is revolutionary in solving the position update for a vehicle. Most common solution is the usage of acoustic transponders and triangulation algorithm as already exposed in Chapter \S~\ref{chap:capabilities}. Williams uses a priori elevation maps of the sea floor, recorded by ships. Depth information obtained from such mapping is assisting the localization process. Localization uses Monte Carlo methods, particularly PF, to manage map-based localization with position and velocity kept within state vector. Non-Gaussian estimates obtained using particles are bounded using depth and altitude observations by using range information to rule out less probable particles. Update is accomplished by resampling the particle distribution with respect to likeliness that the observation detected is received, given the sample of the state space \cite{williams06}. Apart form dealing with non-Gaussian estimates, advantages of such method include ability to track more than one possible target placement, which proves to be useful feature when handling map-based information.  

Eustice \cite{eustice05exactly}.
%%%%%%%%% OTHER THAN TIME-OF-FLY %%%%%%%%%%
%%%%%%%%% SLAM %%%%%%%%%%%%%
Tena Ruiz investigates terrain aided localization of an AUV from the perspective of Simultaneous Localization and Mapping (SLAM) \cite{ruiz01}. Sonar device is used to sense the environment and its readings are used to find targets located not far from the vehicle. SLAM algorithm uses detected targets together with a vehicle model to simultaneously construct the environment map and localize the vehicle by doing filtering. Multiple Hypothesis Tracking Filter is adjusted to the SLAM framework. Necessary step of matching the sonar images with environment targets was accomplished by extracting the features and associating them with the sonar recordings by calculating a score that expresses the probability of a certain target causing a certain sonar output. 

Eustice addresses the problem of precise localization as a prerequisite for high resolution underwater imaging of large objects placed on the sea-bed \cite{eustice05}. Precise navigation would enable decent coverage of the spacious site of interest which is mission task. Proposed solution uses a vision-based SLAM approach together with vehicle's inertial measurements. !! This paper is not entirely clear, uses slam !!


%%%%%%% COMMUNICATION COOPERATIVE NAV PAPERS %%%%%%%%%%
The dissertation of A. Bahr (\cite{bahr08}) proposes an algorithm particularly suited to the underwater environment. Cooperation in navigation is already available in air or the surface of the Earth. The work focuses on cooperative localization where group of vehicles communicate between each other to accomplish cooperative localization. Stated advantages of such approach is that, apart from having more than one vehicle, no additional infrastructure is necessary \cite{bahr08}. Everything comes down to the usual sensor and communication package already available on vehicles\cite{bahr08}. 