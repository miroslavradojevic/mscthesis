%%%%%%%%%%%%%%%%%%%%%%%%%%%%%%%%%%%%%%%%%%%%%%%%%%%%%%%%%%%%%%%%%%%%%%%%%%%
% This is a sample header for a sample dissertation. Fill in the name,
% and the other information. LaTeX will work out the table of
% content, the list of figures and of tables for you.
%%%%%%%%%%%%%%%%%%%%%%%%%%%%%%%%%%%%%%%%%%%%%%%%%%%%%%%%%%%%%%%%%%%%%%%%%%%

\newpage
\thispagestyle{empty}

% ******* Title page *******
% **************************

\vspace*{2cm}
\begin{center}
{\Large\bf MSc. Thesis DRAFT - Underwater vehicle localisation using Extended Kalman Filter\\} \vspace{2cm} {\large
Author: Miroslav Radojevi\'c \\
Supervisor: Yvan Petillot    \\
\vspace{2cm}
Ocean Systems Lab \\
Heriot-Watt University, Edinburgh (UK)   
}

\end{center}

\vspace{7cm}
\begin{center}
{\large A Thesis Submitted for the Degree of \\MSc Erasmus Mundus
in Vision and Robotics (VIBOT) \\\vspace{0.3cm} $\cdot$ 2011
$\cdot$}
\end{center}
\singlespacing


%ABSTRACT
\begin{abstract}
In order to accomplish various missions, autonomous underwater vehicles need to be capable of estimating their position within the environment. This is a prerequisite of a successful mission since further tasks that need to be achieved strongly rely on navigation information as a source of valuable information.

This thesis is a study on application of an algorithm that would accomplish localization of an underwater vehicle using measurements from number of sensors mounted on it. Well known Extended Kalman Filter algorithm approach was initially suggested as a solution for self-localisation using sensor fusion. The work consists of introductory investigation of the topic, theoretical background, overview of the robot localisation capabilities. Introduction summarizes possible methods and obstacles arising from the nature of the underwater environment and sensor features. Second part goes deeper into the matter explaining practical solution for navigation and the details of the implementation, issues needed to be solved. Eventually, an analysis of the results is offered. Preliminary computer simulations and, eventually, the authentic results recorded from the real missions. 

Emphasis is on improving the existing navigation and finding a way to correct known deficiencies, mostly due to erroneous heading measurement. Thesis is intended to report pros and cons of a practical piece of work where a scientific concept was adopted to solve a real task. 
\vspace*{5cm}



%\begin{center}
%\begin{quote}
%\it Research is what I'm doing when I don't know what I'm
%doing.\,\ldots
%\end{quote}
%\end{center}
%\hfill{\small Werner von Braun}
\begin{center}
\begin{quote}
\it Prediction is difficult - especially of the future.\,\ldots
\end{quote}
\end{center}
\hfill{\small Attributed to Niels Henrik David Bohr (1885-1962)}

\end{abstract}

\doublespacing

%\pagestyle{empty}
\pagenumbering{roman}
\setcounter{page}{1} \pagestyle{plain}


\tableofcontents

\listoffigures
\listoftables

\chapter*{Acknowledgments}
\addcontentsline{toc}{chapter}
         {\protect\numberline{Acknowledgments\hspace{-96pt}}}

???

\pagestyle{fancy}
