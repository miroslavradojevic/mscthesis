\chapter{Conclusions} \label{chap:conclusions}
The main focus of the work presented in the thesis is practical application of Extended Kalman Filter (EKF) for Ocean System Lab's Nessie AUV navigation module. EKF was designed to estimate the location of an underwater robot by processing real-time inertial and position information obtained from sensors. Furthermore, EKF algorithm was utilized as a tool for accomplishing sensor fusion - blending together measurements from different sensors as a part of the estimation process. Some principles of implementation for EKF-based localisation that were already available in literature were adopted and modified into a five d.o.f. vehicle model.

One of the issues that were addressed in the thesis was the suitable management of measurement tasks among mounted sensor devices and the role of EKF in correcting deficiencies. Specific case of heading measurement was tested, since this type of angular information is particularly important for the navigation. Usage of magnetic compass for obtaining heading is possible, however needs serious and precise calibration of the compass on the location itself. Advantage would be the possibility to make corrections based on absolute but quite noisy an unstable heading information that can be filtered using EKF tuned to compensate for the magnetic compass inaccuracies. Due to calibration difficulties, most of the experiments use the magnetic compass for initial heading, and EKF successfully appends the accurate heading rate information obtained from gyroscope to compensate for the lack of absolute heading. 

In conclusion, EKF proves to be useful navigation tool with satisfactory navigation performance and several convenient features. Flexible with the number of sensors involved, capable of successfully combining together different sensory information into a location estimate that tends to be optimal with respect to set expectations, or recovering from the missing measurements, outliers, or noise. The issues with nonlinearity have been addressed and the usage of Unscented Kalman Filter (UKF) tested on real data. Implementation of UKF for localisation would improve the accuracy of approximating nonlinearities in EKF at the same computational cost. However, this would have impact on navigation quality in cases when process model is fairly similar to the real robot movement. Simply said - UKF will tune up the emulation of the mathematical equations (prediction). If the prediction mode has an important role in estimation, UKF will contribute. A possibility to test adopted UKF that randomly takes samples is an idea to cope with filtering noisy position information such as LBL's ``mutipathing'' or imprecisions.      

\section{Future work}
Future work on improving localisation performance involves more trials, particularly ones where the true trajectory has been fixed to known landmarks, so that the results of localisation could be thoroughly evaluated with trustful ground truth. Experiments that involve tilted vehicle movements could make an evaluation of the influence of the 5th degree of freedom on localisation. EKF could be modified so that it works with control inputs - which could contribute robustness of the localisation. 

Finally, the problem of correcting the absolute position with LBL information gives space for improvement since the measured position tends to be quite uncertain and prone to different sorts of noise. Solution for rejecting outliers could rely on some version of back-filtering - filtering based on history of received observations.          