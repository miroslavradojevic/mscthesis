\begin{frame}\frametitle{Conclusions}
Presented work reports the design of an navigation module for a real AUV based on Extended Kalman Filter (EKF) .
\begin{itemize}
\item EKF proves to be useful navigation tool with satisfactory navigation performance and several convenient features: \\
{\scriptsize 
\hspace{0.5cm} capable of successfully combining together different sensory information (\textit{sensor fusion}) \\
\hspace{0.5cm} estimate that tends to be optimal with respect to set expectations \\
\hspace{0.5cm} recovering from the missing measurements \\
\hspace{0.5cm} filtering corrupted position information: outliers or signal noise \\
}
%EKF is a great tool since it tries to satisfy the set uncertainty boundaries and fuse all the available information trying to make the most out of it combined together in one mathematical system. %- ``filtration with semantics''
%\\
\item 5DOF \textit{constant velocity} mathematical model for state prediction was introduced 
%\\ One of the issues that were addressed was the s
\item Suitable management of heading measurement was addressed and the role of EKF in correcting deficiencies
%\\
%\item Sensor fusion was explored as mean of improving the estimation
%\\
\item Nonlinearity issues have been investigated with the usage of Unscented Kalman Filter (UKF)
\end{itemize} 
\end{frame}